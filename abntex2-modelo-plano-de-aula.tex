%% abtex2-modelo-artigo.tex, v-1.9.5 laurocesar
%% Copyright 2012-2015 by abnTeX2 group at http://www.abntex.net.br/
%%
%% This work may be distributed and/or modified under the
%% conditions of the LaTeX Project Public License, either version 1.3
%% of this license or (at your option) any later version.
%% The latest version of this license is in
%%   http://www.latex-project.org/lppl.txt
%% and version 1.3 or later is part of all distributions of LaTeX
%% version 2005/12/01 or later.
%%
%% This work has the LPPL maintenance status `maintained'.
%%
%% The Current Maintainer of this work is the abnTeX2 team, led
%% by Lauro César Araujo. Further information are available on
%% http://www.abntex.net.br/
%%
%% This work consists of the files abntex2-modelo-artigo.tex and
%% abntex2-modelo-references.bib
%%

% ------------------------------------------------------------------------
% ------------------------------------------------------------------------
% abnTeX2: Modelo de Artigo Acadêmico em conformidade com
% ABNT NBR 6022:2003: Informação e documentação - Artigo em publicação
% periódica científica impressa - Apresentação
% ------------------------------------------------------------------------
% ------------------------------------------------------------------------

\documentclass[
	% -- opções da classe memoir --
	article,			% indica que é um artigo acadêmico
	12pt,				% tamanho da fonte
	twoside,			% para impressão apenas no verso. Oposto a twoside
	a4paper,			% tamanho do papel.
	% -- opções da classe abntex2 --
	%chapter=TITLE,		% títulos de capítulos convertidos em letras maiúsculas
	%section=TITLE,		% títulos de seções convertidos em letras maiúsculas
	%subsection=TITLE,	% títulos de subseções convertidos em letras maiúsculas
	%subsubsection=TITLE % títulos de subsubseções convertidos em letras maiúsculas
	% -- opções do pacote babel --
	english,			% idioma adicional para hifenização
	brazil,				% o último idioma é o principal do documento
	sumario=tradicional
]{abntex2-modelo-plano-de-aula}


% ---
% PACOTES
% ---

% ---
% Pacotes fundamentais
% ---
\usepackage{lmodern}			% Usa a fonte Latin Modern
\usepackage[T1]{fontenc}		% Selecao de codigos de fonte.
\usepackage[utf8]{inputenc}		% Codificacao do documento (conversão automática dos acentos)
\usepackage{indentfirst}		% Indenta o primeiro parágrafo de cada seção.
\usepackage{nomencl} 			% Lista de simbolos
\usepackage{color}				% Controle das cores
\usepackage{graphicx}			% Inclusão de gráficos
\usepackage{microtype} 			% para melhorias de justificação
\usepackage{datetime}           % Tempo e hora.
\usepackage{epstopdf}
\usepackage{pgf}
% ---

% ---
% Pacotes adicionais, usados apenas no âmbito do Modelo Canônico do abnteX2
% ---
%\usepackage{lipsum}				% para geração de dummy text
% ---

% ---
% Pacotes de citações
% ---
\usepackage[brazilian,hyperpageref]{backref}	 % Paginas com as citações na bibl
\usepackage[alf]{abntex2cite}	% Citações padrão ABNT
% ---

\usepackage{listings}
\usepackage{amsmath}
\usepackage{amssymb}
\usepackage{mathrsfs}
\usepackage{booktabs} % Para Tabelas
\usepackage{subfig}  % permite ter subfiguras
\usepackage{float}
\usepackage{tikz,pgfplots}
\usepackage{pdfpages}
\usepackage{longtable}
\usepackage{framed}

% ---
\newcommand{\sgn}{\mathop{\mathrm{sgn}}}
\DeclareMathOperator{\deriv}{d}



\newcounter{NumberInTable}
\newcommand{\LTNUM}{\stepcounter{NumberInTable}{(\theNumberInTable)}}

\newcommand{\Laplace}[1]{\ensuremath{\mathcal{L}{\left[#1\right]}}}
\newcommand{\InvLap}[1]{\ensuremath{\mathcal{L}^{-1}{\left[#1\right]}}}



% Definição de cores
\definecolor{mygreen}{rgb}{0,0.6,0}
\definecolor{mygray}{rgb}{0.5,0.5,0.5}
\definecolor{mymauve}{rgb}{0.58,0,0.82}
\definecolor{mygraychar}{gray}{0.6}


\definecolor{shadecolor}{rgb}{0.8,0.8,0.8}

\lstset{ %
	aboveskip=3mm,
	belowskip=3mm,
	backgroundcolor=\color{white},   % choose the background color; you must add \usepackage{color} or \usepackage{xcolor}
	basicstyle={\small\ttfamily},        % the size of the fonts that are used for the code
	breakatwhitespace=true,         % sets if automatic breaks should only happen at whitespace
	breaklines=true,                 % sets automatic line breaking
	captionpos=t,                    % sets the caption-position to bottom
	commentstyle=\color{mygreen},    % comment style
	columns=flexible,
	deletekeywords={...},            % if you want to delete keywords from the given language
	escapeinside={\%*}{*)},          % if you want to add LaTeX within your code
	extendedchars=true,              % lets you use non-ASCII characters; for 8-bits encodings only, does not work with UTF-8
	frame=tb,                        % adds a frame around the code
	keepspaces=true,                 % keeps spaces in text, useful for keeping indentation of code (possibly needs columns=flexible)
	keywordstyle=\color{blue},       % keyword style
	language=Matlab,                 % the language of the code
	morekeywords={*,...},            % if you want to add more keywords to the set
	numbers=none,                    % where to put the line-numbers; possible values are (none, left, right)
	numbersep=5pt,                   % how far the line-numbers are from the code
	numberstyle=\tiny\color{mygray}, % the style that is used for the line-numbers
	rulecolor=\color{black},         % if not set, the frame-color may be changed on line-breaks within not-black text (e.g. comments (green here))
	showspaces=false,                % show spaces everywhere adding particular underscores; it overrides 'showstringspaces'
	showstringspaces=false,          % underline spaces within strings only
	showtabs=false,                  % show tabs within strings adding particular underscores
	stepnumber=2,                    % the step between two line-numbers. If it's 1, each line will be numbered
	stringstyle=\color{mymauve},     % string literal style
	tabsize=3,                       % sets default tabsize to 3 spaces
	texcl=true,						 % Permite o uso de acentuação no código
	title=\lstname                   % show the filename of files included with \lstinputlisting; also try caption instead of title
}

%By default, listings does not support multi-byte encoding for source code. The extendedchar option only works for 8-bits encodings such as latin1.
%
%To handle UTF-8, you should tell listings how to interpret the special characters by defining them like so

\iffalse
\lstset{literate=
	{á}{{\'a}}1 {é}{{\'e}}1 {í}{{\'i}}1 {ó}{{\'o}}1 {ú}{{\'u}}1
	{Á}{{\'A}}1 {É}{{\'E}}1 {Í}{{\'I}}1 {Ó}{{\'O}}1 {Ú}{{\'U}}1
	{à}{{\`a}}1 {è}{{\`e}}1 {ì}{{\`i}}1 {ò}{{\`o}}1 {ù}{{\`u}}1
	{À}{{\`A}}1 {È}{{\'E}}1 {Ì}{{\`I}}1 {Ò}{{\`O}}1 {Ù}{{\`U}}1
	{ä}{{\"a}}1 {ë}{{\"e}}1 {ï}{{\"i}}1 {ö}{{\"o}}1 {ü}{{\"u}}1
	{Ä}{{\"A}}1 {Ë}{{\"E}}1 {Ï}{{\"I}}1 {Ö}{{\"O}}1 {Ü}{{\"U}}1
	{â}{{\^a}}1 {ê}{{\^e}}1 {î}{{\^i}}1 {ô}{{\^o}}1 {û}{{\^u}}1
	{Â}{{\^A}}1 {Ê}{{\^E}}1 {Î}{{\^I}}1 {Ô}{{\^O}}1 {Û}{{\^U}}1
	{œ}{{\oe}}1 {Œ}{{\OE}}1 {æ}{{\ae}}1 {Æ}{{\AE}}1 {ß}{{\ss}}1
	{ű}{{\H{u}}}1 {Ű}{{\H{U}}}1 {ő}{{\H{o}}}1 {Ő}{{\H{O}}}1
	{ç}{{\c c}}1 {Ç}{{\c C}}1 {ø}{{\o}}1 {å}{{\r a}}1 {Å}{{\r A}}1
	{€}{{\EUR}}1 {£}{{\pounds}}1 {ã}{{\~a}}1 {õ}{{\~o}}1 {Ã}{{\~A}}1 {Õ}{{\~O}}1
}
\fi
\renewcommand{\lstlistingname}{Código--fonte }% Listing -> Algorithm
\renewcommand{\lstlistlistingname}{Lista de códigos--fonte}% List of Listings -> List of Algorithms

% ---
% Configurações do pacote backref
% Usado sem a opção hyperpageref de backref
\renewcommand{\backrefpagesname}{Citado na(s) página(s):~}
% Texto padrão antes do número das páginas
\renewcommand{\backref}{}
% Define os textos da citação
\renewcommand*{\backrefalt}[4]{
	\ifcase #1 %
	Nenhuma citação no texto.%
	\or
	Citado na página #2.%
	\else
	Citado #1 vezes nas páginas #2.%
	\fi}%
% ---


% ---
% Informações de dados para CAPA e FOLHA DE ROSTO
% ---
\universidade{ Instituto Federal de Minas Gerais}
\centro{Campus São João Evangelista}
 \departamento{Departamento de Desenvolvimento Educacional}
\local{São João Evangelista}
\data{\today}

\autor{Prof. Fernando \textsc{Furtado}}

\tipotrabalho{Notas de Aula}
\disciplina{Filosofia}
\codigo{Fil} % Código da disciplina
\semestre{2021/1}
\aula{Plano de Aula}
\titulo{Desigualdade social no Brasil contemporâneo}

\preambulo{Filosofia}

\email{fernandofurtado@campus.ul.pt}
\telefone{+55 (33) 9 9813 4132}
%\celular{Cel: +55 (33) 9 9813 4132}
%\website{\url{https://www.linkedin.com/in/furtado-fernando/}}
\laboratorio{Departamento de Desenvolvimento Educacional}
\campus{IFMG - MG Campus –- São João Evangelista, Minas}
\turma{ Médio Técnico (Agropecuária, Informática, Nutrição) }
\horario{00:00 -- 00:00}
\sala{sala virtual (MS Teams)}

% ---
% Configurações de aparência do PDF final

% alterando o aspecto da cor azul
\definecolor{blue}{RGB}{41,5,195}

% informações do PDF
\makeatletter
\hypersetup{
     	%pagebackref=true,
		pdftitle={\@title},
		pdfauthor={\@author},
    	pdfsubject={Modelo de Notas de Aulas com abnTeX2},
	    pdfcreator={LaTeX with abnTeX2},
		pdfkeywords={abnt}{latex}{abntex}{abntex2}{notas de aula},
		colorlinks=true,       		% false: boxed links; true: colored links
    	linkcolor=black,          	% color of internal links
    	citecolor=black,        		% color of links to bibliography
    	filecolor=black,      		% color of file links
		urlcolor=black,
		bookmarksdepth=4
}
\makeatother
% ---

% ---
% compila o indice
% ---
\makeindex
% ---

% ---
% Altera as margens padrões
% ---
\setlrmarginsandblock{3cm}{3cm}{*}
\setulmarginsandblock{3cm}{3cm}{*}
\checkandfixthelayout
% ---


% ---
% Espaçamentos entre linhas e parágrafos
% ---

% O tamanho do parágrafo é dado por:
\setlength{\parindent}{1.3cm}

% Controle do espaçamento entre um parágrafo e outro:
\setlength{\parskip}{0.2cm}  % tente também \onelineskip

% Espaçamento simples
\SingleSpacing
\pgfplotsset{compat=1.16}
% ----
% Início do documento
% ----

\usepackage{datapie}

\begin{filecontents}{nota.csv}
Name,Quantity
"AV1",25
"AV2",35
"AV3",40
\end{filecontents}

\DTLloaddb{aval}{nota.csv}
\DTLsetpiesegmentcolor{1}{lightgray}
\DTLsetpiesegmentcolor{2}{mygraychar}
\DTLsetpiesegmentcolor{3}{gray}



\begin{document}

% Seleciona o idioma do documento (conforme pacotes do babel)
%\selectlanguage{english}
\selectlanguage{brazil}

% Retira espaço extra obsoleto entre as frases.
\frenchspacing


%\imprimircapaUFSC

\imprimirletterUFSC

% ]  				% FIM DE ARTIGO EM DUAS COLUNAS
% ---




% ----------------------------------------------------------
% ELEMENTOS TEXTUAIS
% ----------------------------------------------------------
\textual
\pagestyle{notasUFSC}



\begin{snugshade}
	\section{Objetivos} % a serem alcançados pelos alunos e não pelo professor. Podem ser divididos em gerais e específicos.
\end{snugshade}

\subsection{Geral} % projeta resultado geral relativo a execução de conteúdos e procedimentos.
%

Apresentar os aspectos gerais da teoria filosófica para lidar com a questão da desigualdade. Focar no Brasil para estudo de caso. Fazer os estudantes compreenderem conceitos complexos das teorias disponíveis sobre a desigualdade.

Chamar a atenção dos alunos para o enquadramento das discussões sobre a desigualdade no âmbito da Filosofia Política. Mostrar como alguns argumentos filosóficos importantes mostram que desigualdades profundas (como as observadas no Brasil contemporâneo) são injustas. Mostrar como a Teoria da Justiça como Equidade oferece um conceito específico de justiça da distribuição da riqueza em uma sociedade.

Estimular o aluno a pensar criticamente e filosoficamente acerca da justiça da distribuição de riqueza e sobre os problemas causados por desigualdades profundas.

\subsection{Específicos} % especificam resultados esperados observáveis (geralmente de 3 a 4).

\begin{itemize}

	\item Compreender temas e conceitos complexos da área de Filosofia Política e, particularmente, das discussões sobre a justiça da distribuição de riquezas.
	\item Ser capaz de discutir de maneira independente e autônoma os problemas filosóficos cruciais da área.
	%\item Ter condições de pensar a questão dos Movimentos Sociais de maneira independente e autônoma.
	% \item Ter condições de pensar as questões filosóficas relacionadas com a tecnologia de maneira independente e autônoma.
	\item  Escrever um  pequeno ensaio no qual articula com suas próprias palavras as ideias e argumentos centrais que foram discutidos em sala.

\end{itemize}

\newpage


\begin{snugshade}
	\section{Conteúdos} % conteúdos programados para a aula organizados em tópicos (de 4 a 8).
\end{snugshade}
	%Sociologia como estudo da interação humana; cultura e sociedade; os valores sociais; mobilização social e canais de mobilidade; o indivíduo na sociedade; engenharia e sociedade; instituições sociais; sociedade brasileira; mudanças sociais e perspectivas.

\begin{enumerate}
	%\item Estrutura geral da teoria dos Movimentos Sociais


%	\begin{itemize}
		%\item Movimentos Sociais: o que são e quais os seus tipos? %---> inclui definição dos termos
		%\item Há uma definição para \textit{Movimento Social}?
		%\item As principais características dos Movimentos Sociais.
		%\item Movimentos Estruturais vs. Movimentos Conjunturais.
		%\item Movimentos Novos e Antigos.
		%\item Movimentos Conservadores vs. Movimentos Transformadores.
		%\item Como agem os Movimentos Sociais?
		%\item O recente e importante papel da Internet e redes sociais.
		%\item A importância dos Movimentos Sociais para uma cidadania plena.
		%\item Existem Movimentos Sociais puramente virtuais?



		\item Introdução geral à filosofia política.
		\item Exemplos da desigualdade observada no Brasil.
		\item O problema da distribuição de riquezas.
		\item Teoria da justiça da distribuição da riqueza de John Rawls: justiça como equidade.
		\item Argumentos para a teoria da justiça como equidade.
		\item Princípio da oportunidade justa e princípio da diferença.
		\item O experimento mental da Posição Original e o Véu da Ignorância.
		\item Regra do MaxMin.




%		\item \textbf{Vinicius:} Movimentos sociais e ciberativismo (\textit{Anonymous}).  % Anonymous (Vinicius)
%		\item \textbf{Nícolas:} IoT, a privacidade e a liberdade individual. % (Nicolas)
%		\item \textbf{Marcos:} Carros autônomos, inteligência artificial e a responsabilidade (moral e legal). %(Marcos Paulo)
%		\item \textbf{Ariadne:} Liberdade de escolha, autonomia e Big Data. % Ariadne
%		\item \textbf{Isabela:} O conceito de trabalho, o surgimento da ``classe dos inúteis'' e a possibilidade da implementação da renda mínima universal. % Black Mirror Fifteen Million Merits   (Isabela)
%		\item \textbf{Gabriel:} \textit{Internet bots}, redes sociais e democracia. %(Gabriel)
%		\item \textbf{Vinicius:}  O papel da ideia de \textit{Open-source software} na sociedade. \textit{Linux}, etc. % (Vinicius)
%		\item \textbf{Nícolas e Luídson:} \textit{Surveillance}, Big Brother (caso Snowden). % Citizenfour (documentary) (Nicolas)
%		\item \textbf{Marcos:} Economia digital (Fintechs), Bitcoin, \textit{Blockchain} e a possibilidade de uma economia (genuinamente) global. % Algum documentário sobre Bitcoin (Marcos Paulo)
%		\item \textbf{Ariadne:} O caso emblemático de Aaron Swartz. % The Internet's own boy (Ariadne)
%		\item \textbf{Isabela:} A importância do acesso à informação e a garantia de uma ciência livre. (\textit{Sci-Hub, Open Library}, etc.) %(Isabela)
%		\item \textbf{Gabriel e Luídson:} IoT + Bitcoin + \textit{Blockchain} + IA = sociedade  caótica? % The Blockchain And Us (Gabriel)



%	\end{itemize}
\end{enumerate}

%\newpage

\begin{snugshade}
	\section{Procedimentos metodológicos} % estratégias relevantes adotadas para alcançar os objetivos.
\end{snugshade}


Esta aula é pensada para usar estratégia pedagógica mista com dois momentos síncronos e um momento assíncrono. No primeiro momento síncrono, o professor irá expor o tema para os alunos pela primeira vez e incentivará a participação dos alunos na discussão. No momento assíncrono, os alunos irão refletir sobre o tema discutido em sala e terão acesso a uma vídeo aula disponibilizada pelo professor e também a um material escrito. Nesse momento assíncrono, os alunos irão produzir um pequeno ensaio discutindo com suas próprias palavras o tema da desigualdade. O material escrito deverá ser enviado ao professor para avaliação. No momento síncrono seguinte, um momento de revisão, o professor irá incentivar os alunos a participar da discussão com o material estudado durante o momento assíncrono.

Esta aula inicia-se com uma rápida revisão dos temas estudados anteriormente e mostra como eles se relacionam com o tema da aula de hoje.

Na sequência é apresentada uma introdução geral à Filosofia Política que visa mostrar como essa disciplina se enquadra no campo de estudo da Filosofia. Neste momento é feita a explicação de como as discussões acerca das desigualdades aparecem nos debates sobre a justiça da distribuição de riqueza ou justiça distributiva. Neste momento são apresentados fotos e casos que exemplificam o problema da desigualdade no Brasil.

A parte mais conceitual da aula está relacionada com a Teoria da Justiça de Jonh Rawls. Essa parte da aula contará com exemplos de experimentos mentais que facilitam a visualização dos conceitos abstratos e aumentam a chance de aprendizagem do estudante de Ensino Médio. Imagens ilustrativas serão usadas para facilitar a compreensão dos conceitos filosóficos abstratos.

Ao fim da aula, o professor fará uma revisão rápida dos conteúdos estudados e explicará as atividades a serem desenvolvidas no momento assíncrono da aula.

\begin{snugshade}
	\section{Atividade: momento assíncrono} % apresentar uma proposta de atividade para os alunos
\end{snugshade}

No momento assíncrono, os alunos deverão escrever um pequeno ensaio (uma ou duas páginas) onde articulam com suas próprias palavras os conceitos centrais discutidos em sala. Como material de apoio para essa atividade, o professor irá disponibilizar a gravação da aula síncrona, uma vídeo aula adicional e um material escrito sobre o tema. Os alunos poderão escrever seus textos com base em qualquer dos materiais oferecidos.

%A metodologia de ensino inicia-se com a apresentação de exemplos de Movimentos Sociais com os quais os estudantes vão se relacionar e reconhecer.
%Na sequência são apresentadas e brevemente discutidas uma hipótese de definição de Movimentos Sociais que será ainda corroborada com apoio na literatura sociológica.

%A estrutura geral da aula será, depois de apresentada uma definição, analisar os diferentes focos que podem ser tomados para compreender e explicar a teoria geral dos Movimentos Sociais. Assim, por exemplo, os Movimentos Sociais serão caracterizados pela suas intenções (alterar ou manter um estado de coisas), pelas suas estruturas organizacionais (são conjunturais ou estruturais), e pelas suas influências e origens históricas (são novos ou antigos).

%Depois da discussão da caracterização adequada para os Movimentos Sociais, será também brevemente discutida a relação entre os Movimentos Sociais e o conceito de cidadania.

%Serão ainda usadas imagens e fotografias importantes que devem chamar a atenção dos estudantes para a relevância do tema.


%Por fim, os estudantes são convidados a pensar sobre a possibilidade de enquadrar vários movimentos quase puramente virtuais no grupo dos Movimentos Sociais, sempre com base nos conceitos e definições discutidos em sala de aula.

% O curso é dividido em duas partes: i) o professor irá apresentar os conceitos fundamentais da teoria filosófica acerca da noção de tecnologia, como o conceito pode ser caracterizado e quais as alternativas teóricas para a compreensão adequada do conceito. ii) Os alunos irão apresentar seminários sobre temas previamente combinados onde terão a oportunidade de aplicar os conceitos estudados a casos particulares.

%O curso é conduzido de uma maneira geral em forma de seminários. Os alunos serão convidados a apresentar trabalhos em sala de aula na forma de seminário e o professor irá conduzir a discussão de maneira a chamar a atenção dos alunos para os problemas e questões sociais que a tecnologia e, especificamente, a engenharia da computação levantam. São discutidas as potencialidades e possibilidades de aplicação da tecnologia para melhorar a organização social e a vida das pessoas. Ao mesmo tempo, é chamada a atenção dos alunos para riscos que a tecnologia traz no que diz respeito à privacidade, à liberdade individual e até mesmo à democracia.

% Além das apresentações por parte do professor de conceitos chave da teoria filosófica e da apresentação de seminários por parte dos alunos, podem ser apresentados materiais audiovisuais (filmes, séries, documentários, etc.) que têm por objetivo incentivar a discussão aprofundada dos temas propostos.

% Os temas que serão trazidos para discussão em sala de aula serão sempre de interesse direto dos alunos e, sempre que possível, tratados de maneira simples e com uso de vocabulário não-teórico. O mesmo também é esperado dos alunos tanto em seus seminários quanto em seus relatórios obrigatórios.

% Os relatórios serão feitos e avaliados individualmente. Os seminários são feitos em grupo (a definir o tamanho dos grupos). Casos excepcionais de impossibilidade de um aluno não ter condições de trabalhar em grupo serão devidamente analisados pelo professor.

% O professor aceitará, mediante consulta, sugestões de temas de trabalho e discussão pelos alunos.

%\newpage

\begin{snugshade}
	\section{Recursos didáticos} % quadro, giz, retro-projetor, filme, música, quadrinhos, etc.
\end{snugshade}

\begin{itemize}

	\item Computador, Internet, MS Teams, Power Point, \LaTeX, Imagens, Fotografias, Google Jan, Google Forms, Google Doc,.

\end{itemize}

\begin{snugshade}
	\section{Avaliação} % pode ser realizada com diferentes propósitos (diagnóstica, formativa e somativa). Interessante explicitar a atividade avaliativa e os critérios de correção.
\end{snugshade}

% Ao final do curso o aluno deverá estar apto a compreender e explicar as questões fundamentais da filosofia da tecnologia.

%Assim, os alunos são esperados serem capazes e escrever um ensaio curto (entre 4 e 10 páginas) onde são capazes de articular os conceitos fundamentais estudados em sala de aula. Dentre os quais:

% Os alunos serão avaliados de três formas: a primeira corresponde a sua participação e assiduidade e entrega dos relatórios das aulas. A segunda corresponde a uma avaliação presencial sobre os temas discutidos em sala. A terceira corresponde será apresentação em grupo de um seminário em sala de aula. Todos os três componentes da nota final são obrigatórios, assim como a presença nas aulas.

Os alunos serão avaliados continuamente com base na participação em sala de aula (momento síncrono), com base na qualidade do pequeno ensaio produzido e, mais tarde, com base em uma avaliação formativa sobre o assunto discutido nesta aula.

O peso das avaliações é designado na página seguinte:

%Ainda que um aluno não compareça a uma aula, é obrigatória (sempre que possível) a entrega do relatório. Pelo menos 90\% dos relatórios devem obrigatoriamente ser entregues ao professor.

% Os relatórios podem ser entregues em sala ou em até uma semana após a data da aula de referência (preferencialmente, pelo SIGAA).

\vspace{1.5cm}
\newpage
\noindent \textbf{Distribuição do peso dos métodos avaliativos:}\\
PAR: participação nas discussões\\
ENS: qualidade do ensaio produzido\\
AVA: avaliação formativa subsequente

\begin{figure}[htbp]
\centering
\DTLpiechart{variable=\quantity, %
innerlabel={\DTLpiepercent\%},%
outerlabel=\name}{aval}{%
\name=Name,\quantity=Quantity}
\caption{Distribuição da nota}
\end{figure}

%\begin{enumerate}
%	\item O que são Movimentos Sociais?
%	\item Quais os tipos de Movimentos Sociais?
%	\item De que maneira os Movimentos Sociais podem ser caracterizado?
%	\item Existem Movimentos Sociais conservadores?
%	\item Qual a importância da internet para os Movimentos Sociais?

%\end{enumerate}



% Referências bibliográficas
\iffalse % commenting stuff out.
% ---- Mapa das notas -----

% 1 Notas Ariadne
\newcommand{\parOneAri}{8.8}
\newcommand{\parTwoAri}{10}
\newcommand{\reOneAri}{10}
\newcommand{\reTwoAri}{8.3}
\newcommand{\apOneAri}{8.5}
\newcommand{\apTwoAri}{9.0}

% 2 Notas Gabriel
\newcommand{\parOneGab}{10}
\newcommand{\parTwoGab}{10}
\newcommand{\reOneGab}{10}
\newcommand{\reTwoGab}{10}
\newcommand{\apOneGab}{10}
\newcommand{\apTwoGab}{10}

% 3 Notas Isabela
\newcommand{\parOneIsa}{10}
\newcommand{\parTwoIsa}{10}
\newcommand{\reOneIsa}{8.6}
\newcommand{\reTwoIsa}{6.7}
\newcommand{\apOneIsa}{9.0}
\newcommand{\apTwoIsa}{10}


% 4 Notas Luídson
\newcommand{\parOneLu}{10}
\newcommand{\parTwoLu}{10}
\newcommand{\reOneLu}{7.5}
\newcommand{\reTwoLu}{6.7}
\newcommand{\apOneLu}{8.5}
\newcommand{\apTwoLu}{10}

% 5 Notas Marcos
\newcommand{\parOneMar}{10}
\newcommand{\parTwoMar}{8.3}
\newcommand{\reOneMar}{10}
\newcommand{\reTwoMar}{10}
\newcommand{\apOneMar}{9.5}
\newcommand{\apTwoMar}{9.5}

% 6 Notas Nícolas
\newcommand{\parOneNi}{8.8}
\newcommand{\parTwoNi}{5}
\newcommand{\reOneNi}{10}
\newcommand{\reTwoNi}{10}
\newcommand{\apOneNi}{8.5}
\newcommand{\apTwoNi}{8.5}

% 7 Notas Vinícius
\newcommand{\parOneVi}{10}
\newcommand{\parTwoVi}{10}
\newcommand{\reOneVi}{10}
\newcommand{\reTwoVi}{10}
\newcommand{\apOneVi}{8.0}
\newcommand{\apTwoVi}{8.5}


% ----- Código para compilar as notas

\begin{table}[htbp]
	\centering
\begin{tabular}{ c@{}c@{}c@{}c@{}c@{}c@{} }% <-- added @{}
\toprule\toprule
{} \hspace{1cm} & \textbf{Aluno} & \hspace{1cm}\textbf{Av1} & \hspace{1cm}\textbf{Av2} & \hspace{1cm}\textbf{Av3} & \hspace{1cm}\textbf{Total} \\
\toprule
1 \hspace{0.3cm} & Ariadne &
	\hspace{1cm}\begin{tabular}{c c} \pgfmathparse{(\parOneAri * 2.5/2))}\pgfmathresult & \pgfmathparse{(\parTwoAri * 2.5/2))}\pgfmathresult \end{tabular} & % AV1: participação e assiduidade.
	\hspace{1cm}\begin{tabular}{c c} \pgfmathparse{(\reOneAri * 3.5/2))}\pgfmathresult & \pgfmathparse{(\reTwoAri * 3.5/2))}\pgfmathresult \end{tabular} & % AV2: relatórios
	\hspace{1cm}\begin{tabular}{c c} \pgfmathparse{((\apOneAri * 4/2))}\pgfmathresult & \pgfmathparse{((\apTwoAri * 4/2))}\pgfmathresult \end{tabular} &% AV3: apresentação.
	\hspace{1cm}\begin{tabular}{c} \textbf{\pgfmathparse{int(round(((\parOneAri * 2.5/2) + (\parTwoAri* 2.5/2) + ((\reOneAri * 3.5/2) + (\reTwoAri * 3.5/2) + ((\apOneAri * 4/2) + (\apTwoAri *4/2))))}\pgfmathresult} \end{tabular} \\% Total.

2 \hspace{0.3cm} & Gabriel &
	\hspace{1cm}\begin{tabular}{c c} \pgfmathparse{(\parOneGab * 2.5/2))}\pgfmathresult & \pgfmathparse{(\parTwoGab * 2.5/2))}\pgfmathresult \end{tabular} & % AV1: participação e assiduidade.
	\hspace{1cm}\begin{tabular}{c c} \pgfmathparse{(\reOneGab * 3.5/2))}\pgfmathresult & \pgfmathparse{(\reTwoGab * 3.5/2))}\pgfmathresult \end{tabular} & % AV2: relatórios
	\hspace{1cm}\begin{tabular}{c c} \pgfmathparse{((\apOneGab * 4/2))}\pgfmathresult & \pgfmathparse{((\apTwoGab * 4/2))}\pgfmathresult \end{tabular} &% AV3: apresentação.
	\hspace{1cm}\begin{tabular}{c} \textbf{\pgfmathparse{int(round(((\parOneGab * 2.5/2) + (\parTwoGab* 2.5/2) + ((\reOneGab * 3.5/2) + (\reTwoGab * 3.5/2) + ((\apOneGab * 4/2) + (\apTwoGab *4/2))))}\pgfmathresult} \end{tabular} \\% Total.

3 \hspace{0.3cm} & Isabela &
	\hspace{1cm}\begin{tabular}{c c} \pgfmathparse{(\parOneIsa * 2.5/2))}\pgfmathresult & \pgfmathparse{(\parTwoIsa * 2.5/2))}\pgfmathresult \end{tabular} & % AV1: participação e assiduidade.
	\hspace{1cm}\begin{tabular}{c c} \pgfmathparse{(\reOneIsa * 3.5/2))}\pgfmathresult & \pgfmathparse{(\reTwoIsa * 3.5/2))}\pgfmathresult \end{tabular} & % AV2: relatórios
	\hspace{1cm}\begin{tabular}{c c} \pgfmathparse{((\apOneIsa * 4/2))}\pgfmathresult & \pgfmathparse{((\apTwoIsa * 4/2))}\pgfmathresult \end{tabular} &% AV3: apresentação.
	\hspace{1cm}\begin{tabular}{c} \textbf{\pgfmathparse{int(round(((\parOneIsa * 2.5/2) + (\parTwoIsa* 2.5/2) + ((\reOneIsa * 3.5/2) + (\reTwoIsa * 3.5/2) + ((\apOneIsa * 4/2) + (\apTwoIsa *4/2))))}\pgfmathresult} \end{tabular} \\% Total.

4 \hspace{0.3cm} & Luídson &
	\hspace{1cm}\begin{tabular}{c c} \pgfmathparse{(\parOneLu * 2.5/2))}\pgfmathresult & \pgfmathparse{(\parTwoLu * 2.5/2))}\pgfmathresult \end{tabular} & % AV1: participação e assiduidade.
	\hspace{1cm}\begin{tabular}{c c} \pgfmathparse{(\reOneLu * 3.5/2))}\pgfmathresult & \pgfmathparse{(\reTwoLu * 3.5/2))}\pgfmathresult \end{tabular} & % AV2: relatórios
	\hspace{1cm}\begin{tabular}{c c} \pgfmathparse{((\apOneLu * 4/2))}\pgfmathresult & \pgfmathparse{((\apTwoLu * 4/2))}\pgfmathresult \end{tabular} &% AV3: apresentação.
	\hspace{1cm}\begin{tabular}{c} \textbf{\pgfmathparse{int(round(((\parOneLu * 2.5/2) + (\parTwoLu* 2.5/2) + ((\reOneLu * 3.5/2) + (\reTwoLu * 3.5/2) + ((\apOneLu * 4/2) + (\apTwoLu *4/2))))}\pgfmathresult} \end{tabular} \\% Total.

5 \hspace{0.3cm} & Marcos &
	\hspace{1cm}\begin{tabular}{c c} \pgfmathparse{(\parOneMar * 2.5/2))}\pgfmathresult & \pgfmathparse{(\parTwoMar * 2.5/2))}\pgfmathresult \end{tabular} & % AV1: participação e assiduidade.
	\hspace{1cm}\begin{tabular}{c c} \pgfmathparse{(\reOneMar * 3.5/2))}\pgfmathresult & \pgfmathparse{(\reTwoMar * 3.5/2))}\pgfmathresult \end{tabular} & % AV2: relatórios
	\hspace{1cm}\begin{tabular}{c c} \pgfmathparse{((\apOneMar * 4/2))}\pgfmathresult & \pgfmathparse{((\apTwoMar * 4/2))}\pgfmathresult \end{tabular} &% AV3: apresentação.
	\hspace{1cm}\begin{tabular}{c} \textbf{\pgfmathparse{int(round(((\parOneMar * 2.5/2) + (\parTwoMar* 2.5/2) + ((\reOneMar * 3.5/2) + (\reTwoMar * 3.5/2) + ((\apOneMar * 4/2) + (\apTwoMar * 4/2))))}\pgfmathresult} \end{tabular} \\% Total.

6 \hspace{0.3cm} & Nícolas &
	\hspace{1cm}\begin{tabular}{c c} \pgfmathparse{(\parOneNi * 2.5/2))}\pgfmathresult & \pgfmathparse{(\parTwoNi * 2.5/2))}\pgfmathresult \end{tabular} & % AV1: participação e assiduidade.
	\hspace{1cm}\begin{tabular}{c c} \pgfmathparse{(\reOneNi * 3.5/2))}\pgfmathresult & \pgfmathparse{(\reTwoNi * 3.5/2))}\pgfmathresult \end{tabular} & % AV2: relatórios
	\hspace{1cm}\begin{tabular}{c c} \pgfmathparse{((\apOneNi * 4/2))}\pgfmathresult & \pgfmathparse{((\apTwoNi * 4/2))}\pgfmathresult \end{tabular} &% AV3: apresentação.
	\hspace{1cm}\begin{tabular}{c} \textbf{\pgfmathparse{int(round(((\parOneNi * 2.5/2) + (\parTwoNi* 2.5/2) + ((\reOneNi * 3.5/2) + (\reTwoNi * 3.5/2) + ((\apOneNi * 4/2) + (\apTwoNi *4/2))))}\pgfmathresult} \end{tabular} \\% Total.

7 \hspace{0.3cm} & Vinícius &
		\hspace{1cm}\begin{tabular}{c c} \pgfmathparse{(\parOneVi * 2.5/2))}\pgfmathresult & \pgfmathparse{(\parTwoVi * 2.5/2))}\pgfmathresult \end{tabular} & % AV1: participação e assiduidade.
		\hspace{1cm}\begin{tabular}{c c} \pgfmathparse{(\reOneVi * 3.5/2))}\pgfmathresult & \pgfmathparse{(\reTwoVi * 3.5/2))}\pgfmathresult \end{tabular} & % AV2: relatórios
		\hspace{1cm}\begin{tabular}{c c} \pgfmathparse{((\apOneVi * 4/2))}\pgfmathresult & \pgfmathparse{((\apTwoVi * 4/2))}\pgfmathresult \end{tabular} &% AV3: apresentação.
		\hspace{1cm}\begin{tabular}{c} \textbf{\pgfmathparse{int(round(((\parOneVi * 2.5/2) + (\parTwoVi* 2.5/2) + ((\reOneVi * 3.5/2) + (\reTwoVi * 3.5/2) + ((\apOneVi * 4/2) + (\apTwoVi *4/2))))}\pgfmathresult} \end{tabular} \\% Total.

\bottomrule\bottomrule
\end{tabular}

\caption{Mapa de avaliações}
\label{notas}
\end{table}


\fi

%\pagestyle{plain}
%\begin{thebibliography}{}


	%\bibitem{AVRITZER} AVRITZER, Leonardo. Sociedade civil e participação no Brasil democrático. In: Experiências nacionais de participação social. São Paulo: Cortez, 2009, p. 27-54. (Coleção Democracia Participativa).

	%\bibitem{CARVALHO} CARVALHO, José Murilo. Cidadania no Brasil: o longo caminho. 5. ed. Rio de Janeiro: Civilização Brasileira, 2004.

	%\bibitem{FRANK} FRANK, André Gunder; FUENTES, Marta. Dez teses acerca dos Movimentos Sociais. Lua Nova, São Paulo, nº 17, junho 1989.

	%\bibitem{GOHN} GOHN, Maria da Glória. Sociologia dos Movimentos Sociais. 2. ed. São Paulo: Cortez Editora, 2014. (Questões da nossa época, 47).

	%\bibitem{GOHNa} GOHN, Maria da Glória. Movimentos Sociais na Contemporaneidade. Revista Brasileira de Educação, Minas Gerais, v.16, n. 47, p. 333-351, maio/ago. 2011.

	%\bibitem{HONNETH} HONNETH, Axel. Luta por reconhecimento: a gramática moral dos conflitos sociais. Tradução. Luiz Repa. São Paulo: Ed. 34, 2003.

	%\bibitem{SCHERER} SCHERER-WARREN, Ilse. Movimentos Sociais no Brasil contemporâneo. História: Debates e Tendências, vol. 7, n 1, p. 9--21, jan.--jun. 2008.

	%\bibitem{SCHERERK} SCHERER-WARREN, Ilse; KRISCHKE, P. (orgs.). Uma revolução no quotidiano?: os novos Movimentos Sociais na América do Sul. São Paulo: Brasiliense, 1987

%\end{thebibliography}






% ----------------------------------------------------------
% ELEMENTOS PÓS-TEXTUAIS
% ----------------------------------------------------------
% \postextual



\end{document}
